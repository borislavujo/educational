\chapter*{Abstract}

% rare events are everywhere
% 
This work presents a new method for calculating rate constants for configurational transitions described in terms of a master equation.
The method is based on constraining molecular dynamics simulations to boxes in configuration space, and is also known as ``boxed molecular dynamics".
% Rate constants can be easily calculated even for systems deviating from an exponential distribution, as a result of the presence of internal barriers and the roughness of dividing surfaces.
Rate constants can be easily calculated even for systems deviating from an exponential distribution of the first passage times, as a result of the presence of internal barriers and roughness of the dividing surfaces.
The theoretical justification of the method is based on the concept of mean first passage times.
One of the assumptions of the reactive flux formulation is omitted; regression of the population evolution is used instead of calculation of the rate constant at a single point, so the distribution of first passage times is not required to be strictly exponential.
% and superiority over reactive flux method is illustrated on toy models.
The efficiency and correctness of the new method, boxed molecular dynamics in the first passage time formulation of the rate constants (FPT-BXD), is demonstrated for toy models.
%The method is currently under development to be applicable to more complex systems
Preliminary results of simulations for a cluster of Lennard-Jones discs using FPT-BXD are discussed.
