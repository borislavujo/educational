\section{Master Equations}

A dynamical system\cite{Holodniok1986} is a tuple \(\left\{ {\bf S} , \varphi \right\}\), where ${\bf S}$ is the phase space and $\varphi$ is a mapping \(\varphi : {\bf S} \times \mathbb{R} \rightarrow {\bf S} \) satisfying two conditions:
\begin{equation}
\label{eq:dyn-conds}
\begin{aligned}
\begin{split}
\varphi({\bf x}, 0) & = {\bf x} & \forall {\bf x} \in {\bf S} &
\\
\varphi(\varphi({\bf x},t), s) & = \varphi({\bf x}, t+s) ~~~ & \forall t,s \in \mathbb{R}, & ~ \forall {\bf x} \in {\bf S} ~.
\end{split}
\end{aligned}
\end{equation}
The physical meaning of the real number $t$ in equation (\ref{eq:dyn-conds}) is the evolution time between states ${\bf x}$ and $\varphi({\bf x},t)$.
The first passage time (FPT) can be defined for each point ${\bf x}$ in ${\bf S}$ and a set of points ${\rm C} \subset {\bf S}$ as the minimum positive value of time $t^{fp}_{\rm C}$ satisfying the condition
\begin{equation}
\label{eq:dynamika}
\varphi({\bf x}, t^{fp}_{\rm C}) \in {\rm C} ~.
\end{equation}

The ultimate goal of studies in dynamics is to find an approximation of $\varphi$ that is accurate and easy to evaluate.
In the case of classical molecular systems, the phase space ${\bf S}$ is a product of an $N$-dimensional momentum and an $N$-dimensional configuration space. 
The dynamics $\varphi$ (called also the phase flow) are given by a Hamiltonian vector field determined by the PES, generating an autonomous system of 2$N$ ordinary differential equations (SODE):
\begin{equation}
\label{eq:vect-field}
\begin{aligned}
\begin{split}
\dt{q}_i & = \frac{p_i}{m_i}
\\
\dt{p}_i & = - \frac{\partial \mathscr{H}({\bf q},{\bf p})}{\partial q_i} ~,
\end{split}
\end{aligned}
\end{equation}
where $q_i$ and $p_i$ are the $i^{\rm th}$ components of an $N$-dimensional spatial coordinate vector ${\bf p}$ and an $N$-dimensional momentum vector ${\bf p}$, respectively. $m_i$ is the mass of the $i^{\rm th}$ particle.
The Hamiltonian $\mathscr{H}$ generally consists of a non-linear function of coordinates ${\bf q}$ (potential energy $\mathscr{V}({\bf q})$) and the kinetic energy $\mathscr{T}({\bf p}) = \sum_i p_i^2 /2 m_i$.
Evaluation of phase flow in constant time is possible only exceptionally for such a system.
The dynamics are usually simulated by a numerical integration of the SODE (\ref{eq:vect-field}), which scales linearly with the length and the number of the simulated trajectories.

A common approach to simplify Equation (\ref{eq:vect-field}) is to discretise the phase space into boxes.
In chemistry, this approach is widely used.
Configuration space boxes correspond to species (molecules, ions, radicals etc.) and kinetic equations describe the evolution of the populations in the boxes.
If only unimolecular reactions occur, which is the case for structural transitions, the system (\ref{eq:vect-field}) of kinetic equations reduces to a linear homogeneous SODE:
\begin{equation}
\label{eq:linear-sode}
\dt{\bf x} = {\bf A} {\bf x} ~,
\end{equation}
where each component of vector ${\bf x}$, $x_i$, is a population of the $i^{\rm th}$ box and ${\bf A}$ is the transition matrix specific to the system and the discretisation.
Equation (\ref{eq:linear-sode}) has an analytical solution
\begin{equation}
\label{eq:sode-sol}
{\bf x}(t) = e^{\displaystyle{\bf A} t} ~ {\bf x}(0) ~.
\end{equation}
The right-hand side of equation (\ref{eq:sode-sol}) can be evaluated accurately and quickly for reasonably large systems (up to millions of boxes).

In the context of conformational transitions, equation (\ref{eq:linear-sode}) is called the master equation.\cite{Gillespie1992, Pande2010b}
If the space is divided into $n$ boxes, information about the dynamical behaviour of the system is reduced to $n^2-n$ elements of ${\bf A}$, also called the rate constants.
The answer to the question of {\bf how} the process occurs is given by the boxes undergoing a significant population change during the process.
In chemistry, the set of discrete paths through the boxes is called the reaction mechanism.
The question of ${\bf how ~ fast}$ the process occurs is answered by the value of the overall rate constant, which has to be evaluated by a simulation of the system or using other approximations.
Models for studying the evolution of the populations in the configuration space boxes are special cases of Markov state models (MSM) and have been recently successfully used for studying biomolecules.
\cite{Kasson2006, Kelley2008}

Discretisation of the phase space is a good approximation if the transition from one box to another follows exponential kinetics.
In phase space, such behaviour implies an exponential distribution of the time lengths of the reactive trajectories.
This distribution usually applies if the species are separated by high energy barriers.
However, the presence of high energy barriers dividing the species is neither a necessary nor a sufficient condition.
Another important requirement of the approximation is the decorrelation of input and output trajectories.
The behaviour of the species must be independent on the reaction in which it was produced.\cite{vanKampen1992}
Every discretisation should be checked for the applicability of the MSM,\cite{Chodera2006b} for example by comparing the internal equilibration time\cite{Chodera2007} with the characteristic time of the transition.
More detailed theory and applications of MSM's to biomolecules can be found in methodology papers\cite{Chodera2006a, Noe2008, Bowman2009a} and recent reviews.\cite{Bowman2009b, Pande2010b}


