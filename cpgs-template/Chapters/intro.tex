\chapter{Introduction}

Life is a non-equilibrium phenomenon.
Biological processes result from complex networks of chemical reactions, diffusion and configurational transitions of molecules or supermolecular complexes.
We are generally interested in {\bf how} and {\bf how fast} a particular process occurs.
The question of {\bf how} the process occurs stands for a qualitative information about the pathway and intermediates, which if modified, cause the nature and the rate of the process to change significantly.
The question of {\bf how fast} the process occurs concerns the quantitative description of the dynamics.

An example of an interesting biological process is protein folding, which is nature's solution to an NP-hard\cite{Wales1999} (in some simplified formulations NP-complete\cite{Paterson1996, Crescenzi1998}) complex non-linear optimisation problem.
However, even simplified computer simulations on time scales of seconds using classical molecular dynamics (MD) would take thousands of years with modern computers.
Simulation methods for more efficient simulations of dynamics of molecular systems have to be developed to make studying complex molecular systems feasible.
In the last few decades, we have seen significant developments in methodology for simulating configurational transitions of molecular systems, ranging from small Lennard-Jones clusters to large biomolecules.
Most of the methods are based on the reactive flux approach and transition state theory (TST) developed 80 years ago for chemical reactions, application of which to soft matter with low barriers results in systematic errors.

In the present work, the classical dynamics determined by the potential energy surface (PES)\cite{Wales2003} is studied numerically.
Species are defined as regions on the PES and the rates are defined based on the length of trajectories in the phase space.
This chapter starts from a general formulation of deterministic dynamics of finite-dimensional systems and follows the approximations and the development leading to the previous formulation of boxed molecular dynamics (BXD).\cite{Glowacki2009}


