Another approach to compute rate constants is the calculation of the mean first passage times (MFPT's).
Instead of studying the equilibrium flux, actual trajectories and their evolution times are studied.
Bunker and Hase studied the distribution of FPT's (in their terminology ``gap times")\cite{Bunker1973} and showed that even chemical reactions do not follow strictly exponential distributions.
In the microcanonical ensemble, there is a non-negligible ensemble of periodic trajectories that do not escape from their box.
Behaviour deviating from the statistical RRKM description was studied by plotting histograms of FPT's by Hase and co-workers.\cite{Lourderaj2009}
However, most of the development of the MFPT approach has been considered in configuration space, the MFPT being the solution to a partial differential equation derived from the Smoluchowski equation.\cite{Hanggi1990}
The reciprocal of MFPT (in the configuration space formulation) and the TST rate constant were shown to be equivalent for the high barriers and well-defined dividing surfaces.\cite{Hanggi1990, Muller1997}

