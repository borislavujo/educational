\section{Rate Constant Calculation}

The scientific development of the theory of chemical dynamics dates back to the $19^{\rm th}$ century when van't Hoff studied the dependence of reaction rate on temperature \cite{vantHoff1884} and Arrhenius introduced the concept of activation energy.\cite{Arrhenius1889}
In the following 50 years, great advances were achieved.
Farkas first used the concept of equilibrium flux to calculate reaction rates.\cite{Farkas1927}
Eyring introduced the concept of an ``activated complex",\cite{Eyring1935} the saddle point on the PES connecting the reactant and the product.
He derived a formula for the ``absolute" rate constant for a reaction of any order
\begin{equation}
\label{eq:eyring}
k^{\rm Eyring}_{\rm A\rightarrow B}(T) = \kappa \frac{1}{\beta h} \frac{Z^{\ddagger}}{Z} e^{- \beta E_A} ~,
\end{equation}
where $\kappa$ is the transmission coefficient, an {\it ad hoc} parameter being generally about unity, and $Z^{\ddagger}$ and $Z$ are the partition sums of the activated state and the reactant, respectively. $\beta = 1 / ( k_B T )$ where $T$ is the thermodynamic temperature and $k_B$ is the Boltzmann constant, $h$ is Planck's constant and $E_A$ is the activation energy.
The reactant, activated complex and product are explicitly defined as single points on the PES.

Transition state theory\cite{Polanyi1920, Wigner1937, Wigner1939} (TST) provides the fundamental basis for the reactive flux method used predominantly today.
In TST, the rate constant is defined as the equilibrium flux through the dividing surface divided by the population inside the reactant box.
In 1938, Wigner summarised\cite{Wigner1938} the assumptions of TST:
\begin{enumerate}
\item the adiabatic separation of the movements of the electrons and nuclei (the \mbox{Born-Oppenheimer\cite{Born1927}} approximation),
\item the motion of the nuclei can be described by classical mechanics,
\item all trajectories crossing the dividing surface are reactive (no recrossing of the dividing surface occurs).
\end{enumerate}
Transition state theory is inherently a classical mechanical theory applicable for reactions in which a transition over a state with high energy is the determining \mbox{step.\cite{Wigner1937}}
From Wigner's paper,\cite{Wigner1937} it can be inferred that he does not define species unambiguously by the dividing surface.
The equilibrium rate constant can be calculated as an integral over the dividing surface in phase space satisfying the non-recrossing condition:
\begin{equation}
\label{eq:wigner}
k^{\rm TST}_{\rm A\rightarrow B}(T) = \frac{W^2}{Z} \int \frac{{\rm d} \mathscr{H}_0 ({\bf q},{\bf p}) / {\rm d} t}{|\nabla \mathscr{H}_0 ({\bf q},{\bf p}) |} {\rm d} s ~,
\end{equation}
where $\mathscr{H}_0 ({\bf q},{\bf p}) = 0$ defines the surface and ${\rm d} \mathscr{H}_0 ({\bf q},{\bf p})/ {\rm d} t$ is
\begin{equation}
\label{eq:diff-H}
\begin{split}
\frac{{\rm d} \mathscr{H}_0 ({\bf q},{\bf p})}{{\rm d} t}  & = \sum_i \left( \frac{\partial \mathscr{H}_0 ({\bf q},{\bf p})}{\partial q_i} \frac{\partial (\mathscr{H}({\bf q},{\bf p}) - \mathscr{H}_0 ({\bf q},{\bf p}))}{\partial p_i} - \right.
\\
& \left. - \frac{\partial \mathscr{H}_0 ({\bf q},{\bf p})}{\partial p_i} \frac{\partial (\mathscr{H}({\bf q},{\bf p}) - \mathscr{H}_0 ({\bf q},{\bf p}))}{\partial q_i} \right) ~. \\
\end{split}
\end{equation}
The ``total volume" $W$ in expression (\ref{eq:wigner}) is used to scale the rate constant which Wigner derived for a trimolecular reaction.

The classical TST rate constant in the microcanonical ensemble was developed by Rice,\cite{Rice1927} Ramsperger,\cite{Rice1928} Kassel\cite{Kassel1928} and Marcus\cite{Marcus1956} (RRKM).
The microcanonical rate constant for transitions from A to B, can be written as
\begin{equation}
\label{eq:rrkm}
k^{\rm TST}_{\rm A\rightarrow B}(E) = \frac{g(E)}{h ~ \Omega_{\rm A}(E)} ~ ,
\end{equation}
where $E$ is the total energy, $\Omega_{\rm A}$ is the density of states of box A, and $g(E)$ is defined as
\begin{equation}
\label{eq:g}
g(E) = \int_{V^\ddagger}^E \!\! \Omega^\ddagger (E^\prime) \rm d E^{\prime} ~ ,
\end{equation}
where $\Omega^\ddagger (E^\prime)$ is the density of states at the dividing surface
and $V^\ddagger$ is the minimum potential energy of the transition state ensemble.
From the relationship between the microcanonical and canonical ensembles it follows that $k(T)$ is the Laplace transform of $k(E)$.

The third assumption was soon identified as the main cause of the divergence between the TST rate constants and the rate constants obtained from experiments.
Chandler reformulated the rate constant\cite{Chandler1978} in the formalism of correlation \mbox{functions\cite{Kubo1957}} using Onsager's hypothesis.\cite{Onsager1931a, Onsager1931b}
The simulation method based on this formula is known as the ``Bennett-Chandler" procedure\cite{Frenkel2002} and is usually performed in two steps.
First, the TST rate constant is calculated.
Second, the TST rate constant is corrected by the transmission coefficient $\kappa$:
\begin{equation}
\label{eq:bc-k}
k^{\rm BC}_{\rm A\rightarrow B} = \kappa ~ k^{\rm TST}_{\rm A\rightarrow B} ~.
\end{equation}
$\kappa$ is calculated from the probability of recrossing obtained from simulations of trajectories starting at the dividing surface.

Another approach to correct for recrossings, variational TST,\cite{Truhlar1980, Fast1998} is based on the assumption that the optimum dividing surface is the one that minimises the recrossings.
New insights were brought by studies identifying the transition state ensemble with the hypersurface in the configuration space with the probability to reach products (committor) equal to 0.5,\cite{Pratt1986} and studies of phase space using a normally hyperbolic invariant manifold\cite{Uzer2002, Ezra2009} for construction of the transition state surface.
Kramers studied motion of a Brown particle in a potential field \cite{Kramers1940} and derived analytical formulae for the high and low friction limits.
His results, generalised by Grote and Hynes,\cite{Grote1980} were later shown\cite{Pollak1986} to be equivalent to TST for parabolic barriers.
More information on recent developments of TST can be found in topic reviews.\cite{Truhlar1996, Pollak2005, Vanden-Eijnden2005}

